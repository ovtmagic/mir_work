% This is LLNCS.DEM the demonstration file of
% the LaTeX macro package from Springer-Verlag
% for Lecture Notes in Computer Science,
% version 2.3 for LaTeX2e
%
\documentclass{llncs}
%
% Para cambiar el idioma:
\usepackage[utf8]{inputenc} % el documento esta codificado en utf8
\usepackage[spanish]{babel} % convenciones tipograficas del castellano


\usepackage{makeidx}  % allows for indexgeneration
\usepackage{amssymb}
%\setcounter{tocdepth}{3}
\usepackage{graphicx}
\usepackage[fleqn]{amsmath} % putting displayed equations at a fixed position from the left margin by using the fleqn option for amsmath 

%\usepackage[numbers,sort]{natbib}
%\usepackage{openbib}


\usepackage{url}


%\renewcommand{\thefootnote}{\dag{footnote}}
%
\begin{document}

\title{A Pattern Recognition Approach for Bass Track Identification in MIDI Files.\\ Repetición del
experimento para la melodía y cálculo de errores Tipo 1, Tipo 2 y Tipo 3}
%
\titlerunning{Bass Track Identification in MIDI Files}  % abbreviated title (for running head)
%                                     also used for the TOC unless
%                                     \toctitle is used
%
\author{Octavio Vicente}
%
\authorrunning{Octavio Vicente}   % abbreviated author list (for running head)
%
%%%% list of authors for the TOC (use if author list has to be modified)
\tocauthor{Octavio Vicente}
%
\institute{Universidad de Alicante, Spain,\\
\email{octavio@wipzona.es}\\ 
%WWW home page: \texttt{http://www.ua.es}
}

\maketitle              % typeset the title of the contribution



%\begin{abstract}
%\end{abstract}


%%%%%%%%%%%%%%%%%%%%%%%%%%%%%%%%%%%%%%%%%%%%%%%%%%%%%%%%%%%%%%%%%%%%%%%%%%%%%%%%%%%%%%%%%%%%%%

\section{Introducción}
El objetivo es aplicar el mismo sistema utilizado en la selección de la pista de bajo a la 
selección de la pista que contiene la melodía, y comprobar si se obtienen resultados similares
a los que se obtuvieron en la Tesis de Pierre.

En la segunda parte repito los experimentos de selección de la pista de bajo, obteniendo los
errores de \emph{Tipo 1}, \emph{Tipo 2} y \emph{Tipo 3}. Hay una pequeña desviación en algunos
de los resultados con respecto a los resultados de \emph{paper}. Esto puede ser debido a alguna
corrección que se ha realizado en la obtención de los descriptores.

\section{Melody track selection}

Comparación de los resultados obtenidos en las Tesis de Pierre y los que obtengo yo utilizando el
mismo sistema que en la selección de la pista de bajo. 

\subsection{Melody track selection}

El umbral de error utilizado es $p_{\epsilon}=0.25$, tanto en el experimento realizado, como en
la Tesis de Pierre.

\begin{table}
\begin{center}
\begin{tabular}{ l | l | r | r | r | r }
\hline
   &   Dataset \hspace{0.5cm} &  \hspace{0.5cm} Success & \hspace{0.2cm} Tipo 1 & \hspace{0.2cm} Tipo 2 & \hspace{0.2cm} Tipo 3 \\
\hline
\hline
Melody track selection  &   clas200 	&	 99.5\%	&	 1 	&	 0 	&	 0 	 \\
                        &   jazz200 	&	 98.5\% &	 1 	&	 0 	&	 2 	 \\
                        &   kar200 	&	 83.5\%	&	 10 	&	 3 	&	 20 	 \\
\hline 
Melody track selection  &   CLA200	&	99.0\%	&	2	&	0	&	0	\\
(Tesis Pierre, tabla 3.9)&  JAZZ200	&	99.0\%	&	1	&	0	&	1	\\
                        &   KAR200	&	84.5\%	&	14	&	4	&	13	\\
\hline
\end{tabular}
\caption{Melody track selection results.}
\label{result2}
\end{center}
\end{table}





%%%%%%%%%%%%%%%%%%%%%%%%%%%%%%%%%%%%%%%%%%%%%%%%%%%%%%%%%%%%%%%%%%%%%%%%%%%%%%%%%%%%%%%%%%%%%%

\section{Bass track selection}

Obtención de los errores \emph{Tipo 1}, \emph{Tipo 2} y \emph{Tipo 3} en la selección de la pista de bajo.


\subsection{Bass track selection}


\begin{table}
\begin{center}
\begin{tabular}{ l | l | r | r | r | r }
\hline
   &   Dataset \hspace{0.5cm} &  \hspace{0.5cm} Success & \hspace{0.2cm} Tipo 1 & \hspace{0.2cm} Tipo 2 & \hspace{0.2cm} Tipo 3 \\
\hline
\hline
Bass track selection    &   kar200 	&	 94.5 	&	 5 	&	 5 	&	 1 	 \\
                        &   jazz200 	&	 100.0 	&	 0 	&	 0 	&	 0 	 \\
                        &   clas200 	&	 99.5 	&	 0 	&	 0 	&	 1 	 \\
\hline
\end{tabular}
\caption{Bass track selection results.}
\label{result2}
\end{center}
\end{table}



\subsection{Bass track selection across styles}
\begin{table}
\begin{center}
\begin{tabular}{ l | l | l | r | r | r | r }
\hline
& Training Dataset  & Test Dataset    &  Success & Tipo1 & Tipo 2 & Tipo 3 \\
\hline
\hline
Bass track     &   KAR200 + JAZ200  &   clas200 	& 72.5 	& 0 	& 55 	& 0 	 \\
selection      &   CLA200 + KAR200  &   jazz200 	& 100.0 & 0 	& 0 	& 0 	 \\
               &   CLA200 + JAZ200  &   kar200 		& 91.5 	& 8 	& 4 	& 5 	 \\
\hline
\end{tabular}
\caption{Bass track selection  across styles.}
\label{result3}
\end{center}
\end{table}




%%%%%%%%%%%%%%%%%%%%%%%%%%%%%%%%%%%%%%%%%%%%%%%%%%%%%%%%%%%%%%%%%%%%%%%%%%%%%%%%%%%%%%%%%%%%%%



\end{document}
